\begin{problem}
  The affine transform in heterogeneous coordinates
  is given by
  \begin{align}
    \begin{bmatrix}
      a & b & c \\
      d & e & f \\
      0 & 0 & 1
    \end{bmatrix}~\label{eq:3.1}.
  \end{align}
  Affine transformations are combinations of:
  \begin{enumroman}
    \item Arbitrary linear transformations with $4$ degrees
      of freedom $(a, b, d, e)$.
    \item Translations with $2$ degrees of freedom $(c, f)$.
  \end{enumroman}
  
  \step
  Does affine transformation apply translation first followed by
  arbitrary linear transformation, or the other way around?
  Prove your answer mathematically.

  \step
  \emph{Hint: check if}\\
  $
    \begin{bmatrix}
      a & b & c \\
      d & e & f \\
      0 & 0 & 1
    \end{bmatrix}
    = 
    \begin{bmatrix}
      1 & 0 & c \\
      0 & 1 & f \\
      0 & 0 & 1
    \end{bmatrix}
    \begin{bmatrix}
      a & b & 0 \\
      d & e & 0 \\
      0 & 0 & 1
    \end{bmatrix}
    \quad \text{ or } \quad 
    \begin{bmatrix}
      a & b & c \\
      d & e & f \\
      0 & 0 & 1
    \end{bmatrix}
    =
    \begin{bmatrix}
      a & b & 0 \\
      d & e & 0 \\
      0 & 0 & 1
    \end{bmatrix}
    \begin{bmatrix}
      1 & 0 & c \\
      0 & 1 & f \\
      0 & 0 & 1
    \end{bmatrix}
  $
\end{problem}

\begin{Answer}
  By computing the matrix products, we see that:
  \begin{align}
    \begin{bmatrix}
      1 & 0 & c \\
      0 & 1 & f \\
      0 & 0 & 1
    \end{bmatrix}
    \begin{bmatrix}
      a & b & 0 \\
      d & e & 0 \\
      0 & 0 & 1
    \end{bmatrix}
    &= 
    \begin{bmatrix}
      a & b & c \\
      d & e & f \\
      0 & 0 & 1
    \end{bmatrix}~\label{eq:3.2} \\
    \nonumber \\
    \begin{bmatrix}
      a & b & 0 \\
      d & e & 0 \\
      0 & 0 & 1
    \end{bmatrix}
    \begin{bmatrix}
      1 & 0 & c \\
      0 & 1 & f \\
      0 & 0 & 1
    \end{bmatrix}
    &=
    \begin{bmatrix}
      a & b & ac + bf \\
      d & e & dc + ef \\
      0 & 0 & 1
    \end{bmatrix}
  \end{align}

  \step
  Our desired affine transformation matrix (~\ref{eq:3.1}) matches the first
  matrix product (~\ref{eq:3.2}).

  \step
  Therefore, we achieve the affine transformation by doing
  the equivalent matrix multiplications in order,
  i.e. doing translation first then doing linear transformation.
\end{Answer}
