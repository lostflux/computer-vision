\begin{problem}
  Given a set $N$ of points $\bp_i = (x_i, y_i), i \in \set{1, \ldots, N}$,
  in the image plane, we wish to find the best line passing through those points.

  \begin{enumalph}
    \item One way to solve this problem is to find $(a, b)$ that most closely
      satisfy the equations $y_i = ax_i + b$, in a least-squares sense.
      Write these equations in the form of a \emph{heterogeneous}
      least-squares problem $A \bx = \bb$, where $\bx = (a, b)^T$,
      and give an expression for the least-squares estimate of $\bx$.
      Give a geometric interpretation of the error being minimized,
      and use a simple graph to visualize the error.
      Does this make sense when fitting a line to points in an image?
      \begin{Answer}
        % The least-squares estimate of $\bx$ is given by
        % \[
        %   \bx = (A^T A)^{-1} A^T \bb.
        % \]
        The error being minimized is the sum of the squared distances
        of the points from the line.
        This makes sense when fitting a line to points in an image,
        since the error is the sum of the squared distances of the
        points from the line.

        % draw graph
        % \begin{figure}[H]
        %   \centering
        %   \caption{Error being minimized}
        % \end{figure}
      \end{Answer}

    \newpage
    \item Another way to solve this problem is to find
      $\ell = (a, b, c)$, defined up to scale, that
      most closely satisfies the equations $ax_i + by_i + c = 0$,
      in a least-squares sense.
      Write these equations in the form of a \emph{homogeneous}
      least-squares problem $A \ell = 0$, where $\ell = (a, b, c)^T$
      and $\ell \ne 0$.
      This problem has a trivial solution (zero vector) which is not
      of much use. Describe some ways of avoiding this trivial solution
      and corresponding algorithms for solving the resulting
      optimization problem. Is this approach more or less
      useful than the previous one? Why?
      
      \step
      \emph{Hint: Think about how we can express the distance of a
      point from a line.}
      \begin{Answer}
        % The least-squares estimate of $\ell$ is given by
        % \[
        %   \ell = \textbf{argmin}_{\ell \ne 0} \sum_{i=1}^N \frac{(\ell^T \bp_i)^2}{\ell^T \ell}.
        % \]
        % This problem has a trivial solution (zero vector) which is not
        % of much use. We can avoid this trivial solution by
        % normalizing $\ell$ to have unit length.
        % This approach is more useful than the previous one,
        % since it is more robust to outliers.
      \end{Answer}
  \end{enumalph}
\end{problem}
