\begin{problem}
  Given a set $N$ of points $\bp_i = (x_i, y_i), i \in \set{1, \ldots, N}$,
  in the image plane, we wish to find the best line passing through those points.

  \begin{enumroman}
    \item One way to solve this problem is to find $(a, b)$ that most closely
      satisfy the equations $y_i = ax_i + b$, in a least-squares sense.
      Write these equations in the form of a \emph{heterogeneous}
      least-squares problem $A \bx = \bb$, where $\bx = (a, b)^T$,
      and give an expression for the least-squares estimate of $\bx$.
      Give a geometric interpretation of the error being minimized,
      and use a simple graph to visualize the error.
      Does this make sense when fitting a line to points in an image?
      \begin{answer}
        The set of $N$ points satisfies the following system of equations:
        \begin{align*}
          y_1 &= ax_1 + b \\
          y_2 &= ax_2 + b \\
          \vdots \\
          y_N &= ax_N + b.
        \end{align*}
        % \newpage
        We can rewrite the system as the following heterogeneous
        \emph{least-squares} problem:

        \[
          \underbrace{\begin{bmatrix}
            x_1 & 1 \\
            x_2 & 1 \\
            \vdots & \vdots \\
            x_N & 1
          \end{bmatrix}}_{A}
          \underbrace{\begin{bmatrix}
            a \\
            b
          \end{bmatrix}}_{\bx}
          =
          \underbrace{\begin{bmatrix}
            y_1 \\
            y_2 \\
            \vdots \\
            y_N
          \end{bmatrix}}_{\bb}.
        \]

        To find the least-squares estimate of $\bx$, we solve the
        normal equations $A^T A \bx = A^T \bb$:

        \begin{align*}
          A^T A \bx &= A^T \bb \\
          \underbrace{\begin{bmatrix}
            x_1 & x_2 & \cdots & x_N \\
            1 & 1 & \cdots & 1
          \end{bmatrix}}_{A^T}
          \underbrace{\begin{bmatrix}
            x_1 & 1 \\
            x_2 & 1 \\
            \vdots & \vdots \\
            x_N & 1
          \end{bmatrix}}_{A}
          \underbrace{\begin{bmatrix}
            a \\
            b
          \end{bmatrix}}_{\bx}
          &=
          \underbrace{\begin{bmatrix}
            x_1 & x_2 & \cdots & x_N \\
            1 & 1 & \cdots & 1
          \end{bmatrix}}_{A^T}
          \underbrace{\begin{bmatrix}
            y_1 \\
            y_2 \\
            \vdots \\
            y_N
          \end{bmatrix}}_{\bb} \\ \\
          \underbrace{\begin{bmatrix}
            \sum\limits_{i=1}^N x_i^2 & \sum\limits_{i=1}^N x_i \\
            \sum\limits_{i=1}^N x_i & N
          \end{bmatrix}}_{A^T A}
          \underbrace{\begin{bmatrix}
            a \\
            b
          \end{bmatrix}}_{\bx}
          &=
          \underbrace{\begin{bmatrix}
            \sum\limits_{i=1}^N x_i y_i \\
            \sum\limits_{i=1}^N y_i
          \end{bmatrix}}_{A^T \bb}
        \end{align*}

        Square matrix on the left is invertible,
        ~\footnote{
          If the matrix is not invertible, then the system of equations
          has either no solution (i.e. is inconsistent)
          or infinitely many solutions.
        }
        so we can solve for $\bx$ by left-multiplying both sides
        by its inverse:

        \begin{align*}
          \underbrace{\begin{bmatrix}
            a \\
            b
          \end{bmatrix}}_{\bx}
          &=
          \underbrace{\begin{bmatrix}
            \sum\limits_{i=1}^N x_i^2 & \sum\limits_{i=1}^N x_i \\
            \sum\limits_{i=1}^N x_i & N
          \end{bmatrix}^{-1}}_{(A^T A)^{-1}}
          \underbrace{\begin{bmatrix}
            \sum\limits_{i=1}^N x_i y_i \\
            \sum\limits_{i=1}^N y_i
          \end{bmatrix}}_{A^T \bb}
        \end{align*}

        \newpage
        The error being minimized is the sum of the square
        distances of each point from the line:
        \[ \calE = \sum_{i=1}^N \abs{\abs{y_i - \bx \cdot x_i}}^2. \]

        % draw tikz graph visualizing error
        \begin{figure}[H]
          \centering
          \begin{tikzpicture}[domain=0:10]
            \draw[very thin,color=gray] (-1.1,-1.1) grid (9.9,9.9);

            % initialize total error
            \pgfmathsetmacro{\totalerror}{0}

            % text expression for terms in summation
            \pgfmathsetmacro{\sumexpression}{0}

            % add points and calculate total error
            \foreach \x/\y in {1/4, 2/1, 3/4, 4/3, 5/3, 6/1, 7/6, 8/6, 9/8}{

              %? draw point and offset from line
              % increase the size of the points (twice as big)
              \begin{scope}[every node/.style={scale=1.5}]
                \foreach \x/\y in {1/4, 2/1, 3/4, 4/3, 5/3, 6/1, 7/6, 8/6, 9/8}{
                  \node at (\x, \y) {\textbullet};
                }
              \end{scope}

              \pgfmathsetmacro{\error}{(\y-\x)*(\y-\x)}
              \draw[-, color=red] (\x, \x) -- (\x, \y) node[left] {$\green{\mathbf{\error}}$};
              
              %? concatenate '+ \error' to sum expression
              \xdef\sumexpression{\sumexpression + \error}
              
              %? increment total error
              \pgfmathparse{\totalerror + \error}
              \global\let\totalerror\pgfmathresult
            }

            % display total error
            \node at (5, -2) {$\calE = \sumexpression = \pgfmathprintnumber{\totalerror}$};

            \draw[->] (-0.2,0) -- (10.2,0) node[right] {$x$};
            \draw[->] (0,-1.2) -- (0,10.2) node[above] {$f(x)$};
          
            \draw[color=blue]    plot (\x,\x)   node[right] {$f(x) =x$};

          \end{tikzpicture}
          \caption{Graph of points, fitted line, and error measures.}
        \end{figure}
        Notice that the error associated to each point varies quadratically.
        This is often desirable when fitting a line to points in an image
        since it penalizes points that are farther from the line by a
        greater scale than points that are closer to the line.
        \emph{
          However, if there are outliers in the dataset,
          then using this method to fit a line to points in an image
          can result in one outlier point having a large effect on the
          fitted line and making it less accurate.
        } \\
      \end{answer}

    \newpage
    \item Another way to solve this problem is to find
      $\ell = (a, b, c)$, defined up to scale, that
      most closely satisfies the equations $ax_i + by_i + c = 0$,
      in a least-squares sense.
      Write these equations in the form of a \emph{homogeneous}
      least-squares problem $A \ell = 0$, where $\ell = (a, b, c)^T$
      and $\ell \ne 0$.
      This problem has a trivial solution (zero vector) which is not
      of much use. Describe some ways of avoiding this trivial solution
      and corresponding algorithms for solving the resulting
      optimization problem. Is this approach more or less
      useful than the previous one? Why?
      
      \step
      \emph{Hint: Think about how we can express the distance of a
      point from a line.}
      \begin{answer}
        The set of $N$ points satisfies the following system of equations:
        \begin{align*}
          ax_1 + by_1 + c &= 0 \\
          ax_2 + by_2 + c &= 0 \\
          \vdots \\
          ax_N + by_N + c &= 0.
        \end{align*}
        We can rewrite the system as the following homogeneous
        \emph{least-squares} problem:

        \[
          \underbrace{\begin{bmatrix}
            x_1 & y_1 & 1 \\
            x_2 & y_2 & 1 \\
            \vdots & \vdots & \vdots \\
            x_N & y_N & 1
          \end{bmatrix}}_{A}
          \underbrace{\begin{bmatrix}
            a \\
            b \\
            c
          \end{bmatrix}}_{\ell}
          =
          \underbrace{\begin{bmatrix}
            0 \\
            0 \\
            \vdots \\
            0
          \end{bmatrix}}_{\mathbf{0}}.
        \]

        \newpage
        To find the least-squares estimate of $\ell$, we solve the
        normal equations $A^T A \ell = A^T \mathbf{0}$:

        \begin{align*}
          A^T A \ell &= A^T \mathbf{0} \\
          \underbrace{\begin{bmatrix}
            x_1 & x_2 & \cdots & x_N \\
            y_1 & y_2 & \cdots & y_N \\
            1 & 1 & \cdots & 1
          \end{bmatrix}}_{A^T}
          \underbrace{\begin{bmatrix}
            x_1 & y_1 & 1 \\
            x_2 & y_2 & 1 \\
            \vdots & \vdots & \vdots \\
            x_N & y_N & 1
          \end{bmatrix}}_{A}
          \underbrace{\begin{bmatrix}
            a \\
            b \\
            c
          \end{bmatrix}}_{\ell}
          &=
          \underbrace{\begin{bmatrix}
            x_1 & x_2 & \cdots & x_N \\
            y_1 & y_2 & \cdots & y_N \\
            1 & 1 & \cdots & 1
          \end{bmatrix}}_{A^T}
          \underbrace{\begin{bmatrix}
            0 \\
            0 \\
            \vdots \\
            0
          \end{bmatrix}}_{\mathbf{0}} \\ \\
          \underbrace{\begin{bmatrix}
            \sum\limits_{i=1}^N x_i^2 & \sum\limits_{i=1}^N x_i y_i & \sum\limits_{i=1}^N x_i \\
            \sum\limits_{i=1}^N x_i y_i & \sum\limits_{i=1}^N y_i^2 & \sum\limits_{i=1}^N y_i \\
            \sum\limits_{i=1}^N x_i & \sum\limits_{i=1}^N y_i & N
          \end{bmatrix}}_{A^T A}
          \underbrace{\begin{bmatrix}
            a \\
            b \\
            c
          \end{bmatrix}}_{\ell}
          &=
          \underbrace{\begin{bmatrix}
            0 \\
            0 \\
            0
          \end{bmatrix}}_{A^T \mathbf{0}}
        \end{align*}

        The square matrix to the left is invertible,
        \emph{
          but the vector on the right is the zero vector,
          so multiplying it by the inverse of the matrix
          will not give us useful information about $\ell$.
        }

        Instead, we want to find the \emph{null space} of $A$,
        \[ \nul A = \set{\vec{v} \in \R^3 \mid A \vec{v} = \mathbf{0}}. \]

        \textbf{On finding the null space of $A^TA$:} \\
        We can find a basis for the null space of $A^TA$
        (called the \emph{null basis}) by reducing the matrix
        \[ \begin{bmatrix}
          A^TA & | & \mathbf{0}
        \end{bmatrix}
        \] to reduced row echelon form.
        The null basis will be the set of vectors corresponding
        to the columns of the reduced matrix that do not have
        a pivot column in the original matrix.

        \step
        \textbf{On Comparative Utility of Approaches:}\\
        The second approach is, in a lot of ways, more useful than the first.
        
        \begin{enumarabic}
          \item The right-hand side of the homogeneous least-squares
            problem is the zero vector, so we don't have to worry about
            complicated computation of $A^T \bb$.

          \item The homogeneous approach also gives us a basis for \emph{all}
            vectors in the null space of $A^TA$.
            If there were multiple lines that are best fits for the points
            they they would be easier to find since they would be linear
            combinations of the null basis vectors.
        \end{enumarabic}
      \end{answer}
  \end{enumroman}
\end{problem}
