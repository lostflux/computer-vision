\begin{problem}
  The equation for a conic in the plane using inhomogeneous
  coordinates $(x, y)$ is
  \begin{align}
    ax^2 + bxy + cy^2 + dx + ey + f = 0.~\label{eq:2.1}
  \end{align}

  \begin{enumalph}
    \item Suppose you have a given a set of inhomogeneous points
      $\bx_i = (x_i, y_i)$, $i \in \set{1, \ldots, N}$.
      Derive an expression for the least squares estimate
      of a conic $\bc = (a, b, c, d, e, f)$ passing through those points.
      
      \step
      \emph{
        Note: your expression may take the form of a
        null vector or eigenvector of a matrix.
        If so, you must provide expressions
        for the matrix elements.
      }
      \begin{answer}
        To determine the matching conic, we need to solve the equation
        $A\bx = \mathbf{0}$:
        \begin{align*}
          \underbrace{\begin{bmatrix}
            x_1^2 & x_1y_1 & y_1^2 & x_1 & y_1 & 1 \\
            x_2^2 & x_2y_2 & y_2^2 & x_2 & y_2 & 1 \\
            \vdots & \vdots & \vdots & \vdots & \vdots & \vdots \\
            x_N^2 & x_Ny_N & y_N^2 & x_N & y_N & 1 \\
          \end{bmatrix}}_{\bA}
          \underbrace{\begin{bmatrix}
            a \\
            b \\
            c \\
            d \\
            e \\
            f \\
          \end{bmatrix}}_{\bc}
          &=
          \underbrace{\begin{bmatrix}
            0 \\
            0 \\
            \vdots \\
            0 \\
          \end{bmatrix}}_{\mathbf{0}}
        \end{align*}
        We solve the equation $\bA\bc = 0$ by converting $\bA$ into
        its singular value decomposition
        \[
          A = U \Sigma V^T
        \]
        so that
        \[
          U \Sigma V^T \bx = 0.
        \]

        \step
        From the singular value decomposition, we compute $V$
        (the transpose of $V^T$) and find the column corresponding
        to the smallest singular value.
        This column contains the coefficients of the conic
        fitting the points.




        % We solve for $\bc$ by first multiplying both sides
        % by $A^T$ in order to convert the matrix on the left into a square
        % matrix, then either [1] multiplying both sides by the inverse
        % of the square matrix if the right-hand side is nonzero
        % or [2] finding the null space if the right-hand side is zero.
        % In this case, approach [2] is the desired one:

        % \begin{align*}
        %   \underbrace{\begin{bmatrix}
        %     x_1^2 & x_2^2 & \cdots & x_N^2 \\
        %     x_1y_1 & x_2y_2 & \cdots & x_Ny_N \\
        %     y_1^2 & y_2^2 & \cdots & y_N^2 \\
        %     x_1 & x_2 & \cdots & x_N \\
        %     y_1 & y_2 & \cdots & y_N \\
        %     1 & 1 & \cdots & 1 \\
        %   \end{bmatrix}}_{A^T}
        %   \underbrace{\begin{bmatrix}
        %     x_1^2 & x_1y_1 & y_1^2 & x_1 & y_1 & 1 \\
        %     x_2^2 & x_2y_2 & y_2^2 & x_2 & y_2 & 1 \\
        %     \vdots & \vdots & \vdots & \vdots & \vdots & \vdots \\
        %     x_N^2 & x_Ny_N & y_N^2 & x_N & y_N & 1 \\
        %   \end{bmatrix}}_{A}
        %   \underbrace{\begin{bmatrix}
        %     a \\
        %     b \\
        %     c \\
        %     d \\
        %     e \\
        %     f \\
        %   \end{bmatrix}}_{\bc}
        %   &=
        %   \underbrace{\begin{bmatrix}
        %     0 \\
        %     0 \\
        %     0 \\
        %     0 \\
        %     0 \\
        %     0 \\
        %   \end{bmatrix}}_{A \mathbf{0}}
        % \end{align*}

        % \begin{align*}
        %   \underbrace{\begin{bmatrix}
        %     \sum_{i=1}^N x_i^4 & \sum_{i=1}^N x_i^3y_i & \sum_{i=1}^N x_i^2y_i^2 & \sum_{i=1}^N x_i^3 & \sum_{i=1}^N x_i^2y_i & \sum_{i=1}^N x_i^2 \\
        %     \sum_{i=1}^N x_i^3y_i & \sum_{i=1}^N x_i^2y_i^2 & \sum_{i=1}^N x_iy_i^3 & \sum_{i=1}^N x_i^2y_i & \sum_{i=1}^N x_iy_i^2 & \sum_{i=1}^N x_iy_i \\
        %     \sum_{i=1}^N x_i^2y_i^2 & \sum_{i=1}^N x_iy_i^3 & \sum_{i=1}^N y_i^4 & \sum_{i=1}^N x_iy_i^2 & \sum_{i=1}^N y_i^3 & \sum_{i=1}^N y_i^2 \\
        %     \sum_{i=1}^N x_i^3 & \sum_{i=1}^N x_i^2y_i & \sum_{i=1}^N x_iy_i^2 & \sum_{i=1}^N x_i^2 & \sum_{i=1}^N x_iy_i & \sum_{i=1}^N x_i \\
        %     \sum_{i=1}^N x_i^2y_i & \sum_{i=1}^N x_iy_i^2 & \sum_{i=1}^N y_i^3 & \sum_{i=1}^N x_iy_i & \sum_{i=1}^N y_i^2 & \sum_{i=1}^N y_i \\
        %     \sum_{i=1}^N x_i^2 & \sum_{i=1}^N x_iy_i & \sum_{i=1}^N y_i^2 & \sum_{i=1}^N x_i & \sum_{i=1}^N y_i & \sum_{i=1}^N 1 \\
        %   \end{bmatrix}}_{A^TA}
        %   \underbrace{\begin{bmatrix}
        %     a \\
        %     b \\
        %     c \\
        %     d \\
        %     e \\
        %     f \\
        %   \end{bmatrix}}_{\bc}
        %   &=
        %   \underbrace{\begin{bmatrix}
        %     0 \\
        %     0 \\
        %     0 \\
        %     0 \\
        %     0 \\
        %     0 \\
        %   \end{bmatrix}}_{A \mathbf{0}}
        % \end{align*}
        
        % To solve for $\bc$, we need to find the null space of $A^TA$
        % in the same manner as in Problem 1. part (ii).
      \end{answer}
    \item In general, what is the minimum value of $N$ that
      allows a unique solution for $\bc$?
      \begin{answer}
        The minimum value of $N$ is $5$.

        Since there are five unknowns in $\bc$ ---
        $a, b, c, d, e$ --- we need at least five equations
        to solve for $\bc$.
        Therefore, the minimum value of $N$ is $5$.

        Note that $f$ can
        be moved to the right-hand side of the equation.
      \end{answer}

    \newpage
    \item Homogenize equation ~\ref{eq:2.1} by making the
      substitutions $x \gets x_1/x_3$ and $y \gets y_1/y_3$,
      and show that in terms of homogeneous coordinates
      $(\bx = (x_1, x_2, x_3))$ the conic can be expressed in
      matrix form
      \[ \bx^T\bC\bx = 0, \]
      where $\bC$ is a symmetric matrix.
      \begin{answer}
        By substituting $x \gets x_1/x_3$ and $y \gets y_1/y_3$,
        and simplifying the resulting equation, we get:
        \begin{align*}
          ax^2 + bxy + cy^2 + dx + ey + f &= 0 \\ \\
          a\parens{\frac{x_1}{x_3}}^2 + b\frac{x_1y_1}{x_3y_3}
          + c\parens{\frac{y_1}{y_3}}^2
          + d\frac{x_1}{x_3} + e\frac{y_1}{y_3} + f &= 0 \\ \\
          \frac{x_1^2}{x_3^2}a + \frac{x_1y_1}{x_3y_3}b
          + \frac{y_1^2}{y_3^2}c
          + \frac{x_1}{x_3}d + \frac{y_1}{y_3}e + f &= 0 \\ \\
          \frac{x_1^2}{x_3^2}(a) + \frac{x_1y_1}{x_3y_3}(b/2 + b/2)
          + \frac{y_1^2}{y_3^2}(c) + \frac{x_1}{x_3}(d/2 + d/2)
          + \frac{y_1}{y_3}(e/2 + e/2) + f &= 0 \\ \\
          \begin{bmatrix}
            (a) \frac{x_1}{x_3} & (b/2) \frac{x_1}{x_3} & (d/2) \frac{x_1}{x_3} \\
            (b/2) \frac{y_1}{y_3} & (c) \frac{y_1}{y_3} & (e/2) \frac{y_1}{y_3} \\
            d/2 & e/2 & f \\
          \end{bmatrix}
          \begin{bmatrix}
            \frac{x_1}{x_3} \\
            \frac{y_1}{y_3} \\
            1 \\
          \end{bmatrix} &= 0 \\ \\
          \begin{bmatrix}
            \frac{x_1}{x_3} & \frac{y_1}{y_3} & 1 \\
          \end{bmatrix}
          \begin{bmatrix}
            a & b/2 & d/2 \\
            b/2 & c & e/2 \\
            d/2 & e/2 & f \\
          \end{bmatrix}
          \begin{bmatrix}
            \frac{x_1}{x_3} \\
            \frac{y_1}{y_3} \\
            1 \\
          \end{bmatrix}
          &= 0 \\ \\
          \underbrace{\begin{bmatrix}
            x & y & 1 \\
          \end{bmatrix}}_{\bx^T}
          \underbrace{\begin{bmatrix}
            a & b/2 & d/2 \\
            b/2 & c & e/2 \\
            d/2 & e/2 & f \\
          \end{bmatrix}}_{\bC}
          \underbrace{\begin{bmatrix}
            x \\
            y \\
            1 \\
          \end{bmatrix}}_{\bx}
          &= 0
        \end{align*}
      \end{answer}

    \newpage
    \item Suppose we apply a projective transformation to our
      points $\bx_i^{'} = \bH\bx_i$.
      The transformed points $x_i'$ will lie on a transformed
      conic represented by a new symmetric matrix $\bC'$.
      Write an equation that specifies the relationship between
      $\bC'$ and $\bC$ in terms of the homography $\bH$.
      \begin{answer}
        \begin{align*}
          \bx^T\bC\bx &= 0 \\
          (\bH\bx)^T\bC(\bH\bx) &= 0 \\
          \bx^T \underbrace{\bH^T\bC\bH}_{\bC'}\bx &= 0 \qquad \text{ (since ${(Ab)^T = b^TA^T}$) } \\
          \intertext{
            The transformed conic satisfies the equation
            $\bx^T\bC'\bx = 0$ where
          }
          \bC' &= \bH^T\bC\bH
        \end{align*}
      \end{answer}
  \end{enumalph}
\end{problem}
