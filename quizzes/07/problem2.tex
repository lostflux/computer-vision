\begin{problem}
  As we discussed in class, the Lambertian and the specular BRDF are the
  two most commonly used reflectance models in physics-based vision.
  \begin{enumroman}
    \item For Lambertian surfaces, the BRDF is a constant function of the input
      and the output directions. For such a material, we often describe the
      reflectance in terms of its \emph{albedo}, which is given the symbol $\rho$.
      For a Lambertian surface, the BRDF and the albedo are related by
      $f_r(\hat{\bv}_i, \hat{\bv}_r) = \rho/\pi$.
      Using conservation of energy, prove that $0 \leq \rho \leq 1$.
      \begin{answer}
        First, note that the albedo is a measure of the fraction of incident
        light that is reflected by the surface. Therefore,
        it cannot be negatie, so $0 \leq \rho$.
        We shall show that $\rho \leq 1$ by using the energy conservation
        principle in reflectance which stipulates that
        the energy reflected by a surface cannot exceed the energy incident
        on the surface. In other words, the integral of the BRDF over the
        hemisphere must be less than or equal to 1:
        \[
          \int_{H^2} f_r(\hat{\bv}_i, \hat{\bv}_r) \cos \theta \d\vec{\omega}_i \leq 1.
        \]
        Substituting in the known values gives:
        \begin{align*}
          \int_{H^2} \frac{\rho}{\pi} \cos \theta \d\vec{\omega}_i &\leq 1 \\
          \frac{\rho}{\pi} \int_{H^2} \frac{\d A \cos\theta }{r^2} &\leq 1
          \qquad\blue{ \parens{\text{since $\d \vec{\omega}_i = \frac{\d A}{r^2}$}}} \\
          \frac{\rho}{\pi} \int_{H^2} \frac{r^2 \cos\theta \sin \theta \d \theta \d \phi}{r^2} &\leq 1
          \qquad\blue{\text{(since $\d A = r^2 \sin \theta \d \theta \d \phi$)}}\\
          \frac{\rho}{\pi} \int_{H^2} \cos\theta \sin \theta \d \theta \d \phi &\leq 1 \\
          \frac{\rho}{\pi} \int\limits_{0}^{2\pi} \int\limits_{0}^{\pi/2} \cos\theta \sin \theta \d \theta \d \phi &\leq 1 \\
          \frac{\rho}{\pi} \int\limits_{0}^{2\pi} \brackets{\frac{\sin^2\theta}{2} }_0^{\pi/2} \d \phi &\leq 1
          \qquad\blue{\parens{\text{since }\int \cos\theta \sin\theta \d \theta = \frac{\sin^2\theta}{2}}} \\
          \frac{\rho}{2\pi} \int\limits_{0}^{2\pi} \d \phi &\leq 1 \\
          \frac{\rho}{2\pi} \cdot \brackets{\phi}_0^{2 \pi} &\leq 1
          \qquad\blue{\parens{\text{since }\int 1 \d \phi = \phi}} \\
          \frac{\rho}{2\pi} \cdot 2\pi &\leq 1 \\
          \rho &\leq 1.
        \end{align*}
        Therefore, \blue{$0 \leq \rho \leq 1$}.
      \end{answer}

    \newpage
    \item A specular surface perfectly reflects \emph{radiance} in the
      \emph{mirror direction}. Concretely, consider a (non-absorbing)
      specular surface patch with normal $\hat{\bn}$.
      For any incident direction $\hat{\bv}_i$, the mirror direction
      equals $\hat{\bv}_s = 2(\hat{\bn}^\top \hat{\bv}_i)\hat{\bn} - \hat{\bv}_i$,
      and $L(\hat{\bv}_s) = L(\hat{\bv}_i)$.
      Given this property, derive an expression for the specular BRDF.
      \begin{answer}
        Since the surface is non-absorbing, the Fresnel term
        \( F_r(\vec{\omega}_i) = 1. \)
        The BRDF of an ideal specular reflection is a Dirac delta function as follows:
        \begin{align*}
          f_r(\bx, \vec{\omega}_i, \vec{\omega}_r) &= F_r(\vec{\omega}_i)
            \frac{\delta(\vec{\omega}_i - R(\vec{\omega}_r, \hat{\bn}))}{\cos \theta_i} \\
          &= F_r(\vec{\omega}_i) \frac{\delta(\vec{\omega}_i - R(\vec{\omega}_r,
            \hat{\bn}))}{\cos \theta_i} \\
          &= \frac{\delta(\hat{\bv}_i - \hat{\bv}_s)\cdot{\abs{\hat{\bv}_i}\abs{\bn}}}{\hat{\bv}_i \cdot \hat{\bn}} \\
          &= \frac{\delta(\hat{\bv}_i - 2(\hat{\bn}^\top \hat{\bv}_i)\hat{\bn}
            + \hat{\bv}_i) \cdot{\abs{\hat{\bv}_i}\abs{\bn}}}{\hat{\bv}_i \cdot \hat{\bn}} \\
          &= \frac{\delta(2\hat{\bv}_i - 2(\hat{\bn}^\top \hat{\bv}_i)\hat{\bn})
            \cdot{\abs{\hat{\bv}_i}\abs{\bn}}}
              {\hat{\bv}_i \cdot \hat{\bn}} \\
          \intertext{If the direction vectors are normalized, then
            $\abs{\hat{\bv}_i} = \abs{\hat{\bv}_s} = 1$. Therefore,}
          &= \blue{\frac{\delta(2\hat{\bv}_i - 2(\hat{\bn}^\top \hat{\bv}_i)\hat{\bn})}
              {\hat{\bv}_i \cdot \hat{\bn}}} \\
        \end{align*}
      \end{answer}
  \end{enumroman}
\end{problem}
