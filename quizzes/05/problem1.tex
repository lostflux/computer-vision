\begin{problem}
  As we discused in class, two cameras are said to form a \emph{rectified pair}
  if their camera coordinate systems differ only by a translation of their origins
  (the camera centers) along a direction that is parallel to either the $x$- or $y$-axis
  of their coordinate systems.

  \begin{enumroman}
    \item Derive an expression for the essential matrix $\bE$ of a rectified pair.
      \begin{answer}
        Let $C1$ and $C2$ be two rectified pair cameras,
        with $\bT = \bmat{t_x \\ t_y \\ t_z}$ as the translation vector from $C1$ to $C2$.
        The essential matrix $\bE \colonequals [\bT]_\times \bR$, with $\bR = \ell_3$
        (since there is no rotation between the two cameras),
        is then given by:
        \begin{align*}
          \bE &= [\bT]_\times \bR \\
              &= \begin{bmatrix}
                0 & -t_z & t_y \\
                t_z & 0 & -t_x \\
                -t_y & t_x & 0
              \end{bmatrix} \begin{bmatrix}
                1 & 0 & 0 \\
                0 & 1 & 0 \\
                0 & 0 & 1
              \end{bmatrix}
              \qquad = \qquad \begin{bmatrix}
                0 & -t_z & t_y \\
                t_z & 0 & -t_x \\
                -t_y & t_x & 0
              \end{bmatrix}
        \end{align*}
        As stated in the problem, the translation will be along the $x$- or the $y$-axis,
        meaning that $t_z = 0$, and either $t_x = 0$ or $t_y = 0$.
        Therefore, the essential matrix $\bE$ will be equivalent to:
        % two columns
        \begin{align*}
          \bE_x = \begin{bmatrix}
            0 & 0 & 0 \\
            0 & 0 & -t_x \\
            0 & t_x & 0
          \end{bmatrix} \qquad \text{ or } \qquad
          \bE_y = \begin{bmatrix}
            0 & 0 & t_y \\
            0 & 0 & 0 \\
            -t_y & 0 & 0
          \end{bmatrix}
        \end{align*}
      \end{answer}
    
      \newpage
    \item Prove that the epipolar lines of a rectified pair are parallel to
      the axis of translation.
      \begin{answer}
        The epipolar lines given by
        $\ell' = \bE \bx$ and $\ell = \bE^\top \bx'$.
        \begin{enumarabic}
          \item For $\bE_x$:
            The epipolar lines are given by
            \begin{align*}
              \ell' &= \bE_x \bmat{x \\ y \\ 1}
                    &= \bmat{0 & 0 & 0 \\ 0 & 0 & -t_x \\ 0 & t_x & 0} \bmat{x \\ y \\ 1}
                    &= \bmat{0 \\ -t_x \\ t_x y}
                    & \equiv \bmat{0 \\ \frac{-t_x}{t_x y}  \\ 1}  \\
              \ell &= \bE_x^\top \bmat{x' \\ y' \\ 1}
                    &= \bmat{0 & 0 & 0 \\ 0 & 0 & t_x \\ 0 & -t_x & 0} \bmat{x' \\ y' \\ 1}
                    &= \bmat{0 \\ t_x \\ -t_x y'}
                    & \equiv \bmat{0 \\ \frac{-t_x}{t_x y'} \\ 1}
            \end{align*}
            Thus the lines are given by
            \[
              \ell'_y = \frac{-t_x}{t_x y} \quad \text{and} \quad \ell_y = \frac{-t_x}{t_x y'}.
            \]
            Since the original points do not change, the epipolar lines are of
            the form $y = k$, where $k$ is a constant, and are therefore parallel to the $x$-axis.

          \item For $\bE_y$:
            The epipolar lines are given by
            \begin{align*}
              \ell' &= \bE_y \bmat{x \\ y \\ 1}
                    &= \bmat{0 & 0 & t_y \\ 0 & 0 & 0 \\ -t_y & 0 & 0} \bmat{x \\ y \\ 1}
                    &= \bmat{t_y \\ 0 \\ -t_y x}
                    & \equiv \bmat{\frac{- t_y}{t_y x} \\ 0 \\ 1} \\
              \ell &= \bE_y^\top \bmat{x' \\ y' \\ 1}
                    &= \bmat{0 & 0 & -t_y \\ 0 & 0 & 0 \\ t_y & 0 & 0} \bmat{x' \\ y' \\ 1}
                    &= \bmat{-t_y \\ 0 \\ t_y x'}
                    & \equiv \bmat{\frac{-t_y}{t_y x'} \\ 0 \\ 1}
            \end{align*}

            Thus the lines are given by
            \[
              \ell'_x = \frac{-t_y}{t_y x} \quad \text{and} \quad \ell_x = \frac{-t_y}{t_y x'}.
            \]
            Since the original points do not change, the epipolar lines are of
            the form $x = k$, where $k$ is a constant, and are therefore parallel to the $y$-axis.

            
        \end{enumarabic}
      \end{answer}
  \end{enumroman}
\end{problem}
