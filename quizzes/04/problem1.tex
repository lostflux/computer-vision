\begin{problem}
  \begin{enumroman}
    \item Prove that there exists a homography $\bH$ that satisfies
      \begin{equation}~\label{eq:1}
        \bx_1 \equiv \bH \bx_2
      \end{equation}
      between the 2D points (in homogeneous coordinates) $\bx_1$ and $\bx_2$
      in the images of a \emph{plane} $\Pi$ captured by two $3 \times 4$ camera
      projection matrices $\bP_1$ and $\bP_2$ respectively.
      The $\equiv$ symbol is equality up to scale.
      \emph{
        Note: A degenerate case happens when the plane $\Pi$ contains both cameras'
        centers, in which case there are infinite choices of $\bH$
        satisfying the above equation.
        You can ignore this special case in your answer.
      }

      \begin{answer}
        Write the points $\bx_1$ and $\bx_2$ in terms of $\bP_1$ and $\bP_2$
        and the real-world 3D point $\bX$:
        \begin{align*}
          \bx_1 &= \bP_1 \bX \implies \bP_1^{-1} \bx_1 = \bX \qquad \text{(left-multiply by $\bP_1^{-1}$)} \\
          \bx_2 &= \bP_2 \bX \implies \bP_2^{-1} \bx_2 = \bX \qquad \text{(left-multiply by $\bP_2^{-1}$)}
        \end{align*}
        Thus, we have:
        \begin{align*}
          \bP_1^{-1} \bx_1 &= \bP_2^{-1} \bx_2 \\
          \intertext{So, we have: }
          \bx_1 &= \bP_1 \bP_2^{-1} \bx_2.
        \end{align*}
        Thus, we can write $\bH = \bP_1 \bP_2^{-1}$ such that $\bx_1 \equiv \bH \bx_2$.

      \end{answer}

    \item Prove that there exists a homography $\bH$ that satisfies equaton~\ref{eq:1}
      given two cameras separated by a pure rotation.
      That is, for camera $1$, $\bx_1 = \bK_1 \bmat{\bI & \bzero} \bX$,
      and for camera $2$, $\bx_2 = \bK_2 \bmat{\bR & \bzero} \bX$.
      Note that $\bK_1$ and $\bK_2$ are the $3 \times 3$ intrinsic matrices
      of the two cameras and are different. $\bI$ is the $3 \times 3$ identity matrix,
      $\bzero$ is the $3 \times 1$ zero vector, and $\bX$ is a point in 3D space.
      $\bR$ is the $3 \times 3$ rotation matrix of the camera.

      \begin{answer}
        First, to simplify the notation, note that:
        \[ \bmat{\bR & \bzero} \bX = \bR \bmat{\bI & \bzero} \bX, \]
        So, define $\bX' = \bmat{\bI & \bzero} \bX$.
        Then, we can write $\bx_2 = \bK_2 \bR \bX'$ and $\bx_1 = \bK_1 \bX'$.
        Therefore;
        \begin{align*}
          \bx_2 &= \bK_2 \bR \bX' \\
                &= \bK_2 \bR \underbrace{\crim{\bK_1^{-1} \bK_1}}_{\bI} \bX' \\
                &= \bK_2 \bR \bK_1^{-1} \underbrace{\bK_1 \bX'}_{\bx_1} \\
                &= \bK_2 \bR \bK_1^{-1} \bx_1
        \end{align*}
        Thus, we can write $\bH = \bK_2 \bR \bK_1^{-1}$
        such that $\bx_2 = \bH \bx_1$.

      \end{answer}

    \newpage
    \item Suppose that a camera is rotating about its center $\bC$,
      keeping the intrinsic parameters $\bK$ constant.
      Let $\bH$ be the homography that maps the view from one camera orientation
      to the view at a second orientation.
      Let $\theta$ be the angle of rotation between the two orientations.
      Show that $\bH^2$ is the homography corresponding to a rotation of $2\theta$.

      \begin{answer}
        Let $\bH_\theta$ be the homography corresponding to a rotation of $\theta$,
        and let $\bH_{2\theta}$ be the homography corresponding to a rotation of $2\theta$.
        As seen above, the homography $\bH_\theta$ is given by $\bH_\theta = \bK \bR(\theta) \bK^{-1}$
        and $\bH_{2\theta} = \bK \bR(2\theta) \bK^{-1}$.

        Let us expand the rotation matrices and find an expression
        for $\bR(2\theta)$ in terms of $\bR(\theta)$.:
        \begin{align*}
          \bR(\theta) &= \bmat{
            \cos \theta & -\sin \theta \\
            \sin \theta & \cos \theta
          } \\
          \bR(2\theta) &= \bmat{
            \cos 2\theta & -\sin 2\theta \\
            \sin 2\theta & \cos 2\theta
          }
        \end{align*}

        Using the double-angle formulae for sine and cosine,
        \[ \sin 2\theta = 2 \sin \theta \cos \theta \]
        \[ \cos 2\theta = \cos^2 \theta - \sin^2 \theta. \]
        Substituting these into the expression for $\bR(2\theta)$ gives
        \begin{align*}
          \bR(2\theta) &= \bmat{
            \cos^2 \theta - \sin^2 \theta & -2 \sin \theta \cos \theta \\
            2 \sin \theta \cos \theta & \cos^2 \theta - \sin^2 \theta
          } \\
          &= \bmat{
            \cos \theta & -\sin \theta \\
            \sin \theta & \cos \theta
          } \bmat{
            \cos \theta & -\sin \theta \\
            \sin \theta & \cos \theta
          } \\
          &= \bR(\theta)^2
        \end{align*}
        Substituting this into the original expression for $\bH_{2\theta}$ gives
        \begin{align*}
          \bH_{2\theta} &= \bK \bR(2\theta) \bK^{-1} \\
          &= \bK \bR(\theta)^2 \bK^{-1} \\
          &= \bK \bR(\theta) \bR(\theta) \bK^{-1} \\
          &= \bK \bR(\theta) \underbrace{\crim{\bK^{-1} \bK}}_{\bI} \bR(\theta) \bK^{-1} \\
          &= \parens{\bK \bR(\theta) \bK^{-1}}\parens{\bK \bR(\theta) \bK^{-1}} \\
          &= \parens{\bK \bR(\theta) \bK^{-1}}^2 \\
          &= \bH_\theta^2
        \end{align*}
      \end{answer}
  \end{enumroman}
\end{problem}
