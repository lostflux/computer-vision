
\begin{problem}
  The continuous convolution of two functions $f(x)$ and $g(x)$
  is given as
  \[
    (f*g)(x) = \int_{-\infty}^{+\infty} f(y)\ g(x-y)\ \d y.
  \]
  \\
  the Gaussian function at scale is defined as
  \[
    G_s(x) = \frac{1}{\sqrt{2\pi s}}\ e^{-\frac{x^2}{2s}},
  \]
  and has the property that
  \[
    \int_{-\infty}^{+\infty} G_s(x) \d x = 1.
  \]
  \\
  Prove that this class of function satisfies the \emph{semigroup property}
   --- the convolution of one Gaussian function with another produces a third
   Gaussian function with scale equal to their sum, i.e.
   \[
    (G_{s_1} * G_{s_2})(x) = G_{s_1 + s_2}(x).
   \]
\end{problem}
\newpage
\begin{Answer}
  Let \[ G_{\alpha}(x) = \frac{1}{\sqrt{2\pi \alpha}}\ e^{(-\frac{x^2}{2\alpha})}, \quad
  G_{\beta}(x) = \frac{1}{\sqrt{2\pi \beta}}\ e^{(-\frac{x^2}{2\beta})} \]
  be Gaussian functions of scale $\alpha$ and $\beta$ respectively.
  \\
  Through direct construction, we see that:
  \begin{align*}
    (G_{\alpha} * G_{\beta})(x) &= \int_{-\infty}^{+\infty} G_{\alpha}(y)\ G_{\beta}(x-y)\ \d y \\
      &= \left(\frac{1}{\sqrt{2\pi \alpha}} 
        \cdot \frac{1}{\sqrt{2\pi \beta}} \right)\ \ 
        \int_{-\infty}^{+\infty} e^{\left(-\frac{y^2}{2\alpha}\right)}\ \cdot e^{\left(-\frac{(x-y)^2}{2\beta}\right)}\ \d y \\
      &= \frac{1}{2\pi \sqrt{\alpha \beta}}\ 
        \int_{-\infty}^{+\infty} e^{ -\frac{y^2}{2\alpha} -\frac{(x-y)^2}{2\beta} }\ \d y \\
    \text{By integrating, we get: } \\
    % \\
    %   &= \frac{1}{2\pi \sqrt{\alpha \beta}}\ \int_{-\infty}^{+\infty} e^{\left( Hy + \frac{x}{2H\beta} \right)^2 - \frac{x^2}{2\beta} - \frac{x^2}{4\beta^2 H^2}}  \d y
    %   \quad \text{where } H = \sqrt{-\frac{1}{2\alpha} - \frac{1}{2\beta}} \\
    &= \frac{1}{\sqrt{2^3\pi (\alpha + \beta)}}\left[ e^{-\frac{x^2}{2(\alpha + \beta)} \erf{\frac{\frac{(\alpha + \beta)y}{\alpha \beta} - x}{\beta \sqrt{\frac{2(\alpha + \beta)}{\alpha \beta}}}}  } \right]_{-\infty}^{\infty} \\
    \\
    \rlap{\text{$\erf{x}$ is an \emph{even} function, and $\erf{x} = 1$ at $x = \infty$.}} \\
    &= \frac{1}{\sqrt{2^3\pi (\alpha + \beta)}} \cdot 2 e^{-\frac{x^2}{2(\alpha + \beta)}} \\
    &= \frac{1}{\sqrt{2\pi (\alpha + \beta)}} \cdot e^{-\frac{x^2}{2(\alpha + \beta)}} \\
    &= G_{\alpha + \beta}(x).\\
    \end{align*}
\end{Answer}
