\begin{problem}
  The continuous convolution of two functions $f(x)$ and $g(x)$
  is given as
  \[
    (f*g)(x) = \int\displaylimits_{-\infty}^{+\infty} f(y)\ g(x-y)\ \d y.
  \]

  \begin{enumroman}
    \item Prove that the convolution of two functions is commutative,
      i.e., changing the order of operands produces the same result.
      \[ (f \ast g) = (g \ast f) \]
      \textit{Hint:} Perform integration by substitution.

      \begin{answer}
        By definition,
        \begin{align*}
          (f \ast g)(x) &= \int\displaylimits_{-\infty}^{+\infty} f(y)\ g(x-y)\ \d y
          \intertext{ Let $u = x - y$, then $\d u = -\d y$ and $y = x - u$.
            Furthermore, when $y = -\infty$, $u = x - (-\infty) \approx +\infty$,
            and when $y = +\infty$, $u = x - (+\infty) \approx -\infty$.  
          }
          (f \ast g)(x) &= \int\displaylimits_{+\infty}^{-\infty} f(x - u)\ g(u)\ (-\d u) \\
                        &= - \int\displaylimits_{+\infty}^{-\infty} f(x - u)\ g(u)\ \d u \\
                        &= \int\displaylimits_{-\infty}^{+\infty} f(x - u)\ g(u)\ \d u \\
                        &= \int\displaylimits_{-\infty}^{+\infty} g(u)\ f(x - u)\ \d u \\
                        &= (g \ast f)(x)
        \end{align*}

      \end{answer}

    \newpage
    \item Prove that the convolution operand is also associative, i.e.,
      rearranging the parentheses on two or more occurrences of the convolution
      operator produces the same result:
      \[ (f \ast g) \ast h = f \ast (g \ast h) \]
      \textit{Hint:} Be careful with variables. Understand which variable should
      be integrated, and why.

      \begin{answer}
        By definition,
        \begin{align*}
          (\phi \ast \zeta)(x)    &= \int\displaylimits_{-\infty}^{+\infty} \phi(y)\ \zeta(x-y)\ \d y.
          \intertext{Plugging in $(f \ast g)$ for $\phi$ and $h$ for $\zeta$, we get: }
          ((f \ast g) \ast h)(x)  &= \int\displaylimits_{-\infty}^{+\infty} (f \ast g)(y)\ h(x-y)\ \d y \\
          \intertext{ Let us expand $(f \ast g)(y)$: }
                                  &= \int\displaylimits_{-\infty}^{+\infty} \parens{\int\displaylimits_{-\infty}^{+\infty} f(z)\ g(y-z)\ \d z}\ h(x-y)\ \d y \\
                                  &= \int\displaylimits_{-\infty}^{+\infty} \int\displaylimits_{-\infty}^{+\infty} f(z)\ g(y-z)\ h(x-y)\ \d z\ \d y \\
          \intertext{ Rearranging the integrals:}
                                  &= \int\displaylimits_{-\infty}^{+\infty} \int\displaylimits_{-\infty}^{+\infty} f(z)\ g(y-z)\ h(x-y)\ \d y\ \d z \\
                                  &= \int\displaylimits_{-\infty}^{+\infty} f(z)\ \parens{\int\displaylimits_{-\infty}^{+\infty} g(y-z)\ h(x-y)\ \d y}\ \d z \\
          \intertext{ To simplify the inner integral, substitute $u = y - z$, then $\d u = \d y$ and $y = z + u$. }
                                  &= \int\displaylimits_{-\infty}^{+\infty} f(z)\ \parens{\int\displaylimits_{-\infty}^{+\infty} g(u)\ h(x - (z + u))\ \d u}\ \d z \\
                                  &= \int\displaylimits_{-\infty}^{+\infty} f(z)\ \parens{\int\displaylimits_{-\infty}^{+\infty} g(u)\ h( (x - z) - u)\ \d u}\ \d z \\
                                  &= \int\displaylimits_{-\infty}^{+\infty} f(z)\ (g \ast h)(x - z)\ \d z \\
                                  &= (f \ast (g \ast h))(x)
        \end{align*}
      \end{answer}
  \end{enumroman}
\end{problem}
