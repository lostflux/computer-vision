\section{Theory Questions}
\label{sec:theory-questions}


Let $\bx_1$ be a set of points in an image and $\bx_2$ be the set of
corresponding points in an image taken by another camera.
Suppose there exists a homography $\bH$ such that:
\[ \bx_1^i \equiv \bH \bx_2^i \qquad (i \in \set{1, 2 \ldots, N}) \]
where $\bx_1^i = \begin{bmatrix} \bx_1^i & \by_1^i & 1 \end{bmatrix}^T$ are in
homogeneous coordinates, $\bx_1^i \in \bx_1$ and $\bH$ is a
$3 \times 3$ matrix.
For each point pair, this relation can be rewritten as
\[ \bA_i \bh = 0 \] where $\bh$ is a column vector reshaped from $\bH$,
and $\bA_i$ is a matrix with elements derived from the points
$\bx_1^i$ and $\bx_2^i$.
This can help calculate $\bH$ from the given point correspondences.

\begin{problem}
  How many degrees of freedom does $\bh$ have?

  \begin{answer}
    $\bh$ has 8 degrees of freedom.
  \end{answer}
\end{problem}

\begin{problem}
  How many points \emph{pair-wise} are needed to solve $\bh$?

  \begin{answer}
    Since we get two equations from each point pair
    (one for each coordinate), we need 4 point pairs to solve $\bh$.
  \end{answer}
\end{problem}

\newpage
\begin{problem}
  Derive $\bA_i$.

  \begin{answer}
    Given $\bx_1^i = \begin{bmatrix} x_1^i & y_1^i & 1 \end{bmatrix}^T$,
    and $\bx_2^i = \begin{bmatrix} x_2^i & y_2^i & 1 \end{bmatrix}^T$, we have:
    \begin{align*}
      \bx_1^i &\equiv \bH \bx_2^i \\
      \begin{bmatrix} x_1^i \\ y_1^i \\ 1 \end{bmatrix} & \equiv
        \begin{bmatrix}
          h_{11} & h_{12} & h_{13} \\ 
          h_{21} & h_{22} & h_{23} \\
          h_{31} & h_{32} & h_{33}
        \end{bmatrix}
        \begin{bmatrix}
          x_2^i \\ y_2^i \\ 1
        \end{bmatrix} \\
      \begin{bmatrix} x_1^i \\ y_1^i \\ 1 \end{bmatrix} & \equiv
        \begin{bmatrix}
          h_{1} x_2^i + h_{2} y_2^i + h_{3} \\
          h_{4} x_2^i + h_{5} y_2^i + h_{6} \\
          h_{7} x_2^i + h_{8} y_2^i + h_{9}
        \end{bmatrix}
      \end{align*}

      Througn rearranging the terms, we can write the equaton as:
      \begin{align*}
        \underbrace{\begin{bmatrix}
          x_2^i & y_2^i & 1 & 0 & 0 & 0 & -x_1^i x_2^i & -x_1^i y_2^i & -x_1^i \\
          0 & 0 & 0 & x_2^i & y_2^i & 1 & -y_1^i x_2^i & -y_1^i y_2^i & -y_1^i
        \end{bmatrix}}_{\bA_i}
        \underbrace{\begin{bmatrix}
          h_{11} \\ 
          h_{12} \\
          h_{13} \\
          h_{21} \\
          h_{22} \\
          h_{23} \\
          h_{31} \\
          h_{32} \\
          h_{33}
        \end{bmatrix}}_{\bh}
        = 
        \underbrace{\begin{bmatrix}
          0 \\
          0
        \end{bmatrix}}_{\bzero}
    \end{align*}
  \end{answer}
\end{problem}

\newpage
\begin{problem}
  When solving $\bA \bh = 0$, in essence you are trying
  to find the $\bh$ that exists in the \emph{null space} of $\bA$.
  What that means is that there would be some non-trivial solution
  for $\bh$ such that that product $\bA \bh$ turns out to be $0$.
  \begin{enumroman}
    \item What will be a trivial solution for $\bh$?
      \begin{answer}
        The vector $\bh = \bzero$ will be a trivial solution.
        While technically a solution to the system $\bA \bh = 0$,
        it is considered ``trivial'' because it does not provide
        any useful information about the homography.
      \end{answer}
    \item Is the matrix $\bA$ full rank? Why, or why not?
      \begin{answer}
        No, the matrix cannot be full-rank.
        This is because the null space of $\bA$ is not $\set{ \bzero }$.
        Essentially, there exists a non-trivial solution to the equation
        $\bA \bh = 0$, and a non-trivial solution implies that the
        matrix $\bA$ is not full rank.
      \end{answer}
    \item What impact will it have on the singular values?
      \begin{answer}
        If $\bA$ is not full rank, then one of its singular values
        will be $0$.
      \end{answer}
    \item What impact will it have on the singular vectors?
      \begin{answer}
        The singular vector corresponding to the singular value $0$
        will be a non-trivial solution to the equation $\bA \bh = 0$,
        or, essentially, the components of the homography vector
        $\bh$.
      \end{answer}
  \end{enumroman}
\end{problem}
