\section{Theory Questions}

\begin{problem}
  Assuming the affine warp model
  $\displaystyle \bW = \begin{bmatrix}
      1 + p_1 & p_3 & p_5 \\ p_2 & 1 + p_4 & p_6 \\ 0 & 0 & 1 \end{bmatrix}$,
  derive the expression for the Jacobian matrix
  $\bJ$ in terms of the warp parameters $\bp = [p_1, p_2, p_3, p_4, p_5, p_6]^\top$.

  \begin{answer}
    The Jacobian matrix $\bJ$ is the matrix of partial derivatives of the warp
    $\bW$ with respect to the warp parameters $\bp$. We have
    \begin{align*}
      \bW &= \begin{bmatrix}
        1 + p_1 & p_3 & p_5 \\ p_2 & 1 + p_4 & p_6 \\ 0 & 0 & 1
      \end{bmatrix} \\
      \bJ &= \begin{bmatrix}
        \frac{\partial \bW}{\partial p_1} & \frac{\partial \bW}{\partial p_2}
        & \frac{\partial \bW}{\partial p_3} & \frac{\partial \bW}{\partial p_4}
        & \frac{\partial \bW}{\partial p_5} & \frac{\partial \bW}{\partial p_6}
      \end{bmatrix}
    \end{align*}
    We can compute the partial derivatives of $\bW$ with respect to the warp
    parameters $\bp$ as follows:
    \begin{alignat*}{5}
      &\frac{\partial \bW}{\partial p_1} &= \begin{bmatrix}
        x \\ 0
      \end{bmatrix}, \qquad
      &&\frac{\partial \bW}{\partial p_2} &= \begin{bmatrix}
        0 \\ x
      \end{bmatrix}, \qquad
      &&\frac{\partial \bW}{\partial p_3} &= \begin{bmatrix}
      y \\ 0
      \end{bmatrix}, \\
      &\frac{\partial \bW}{\partial p_4} &= \begin{bmatrix}
        0 \\ y
      \end{bmatrix}, \qquad
      &&\frac{\partial \bW}{\partial p_5} &= \
      \begin{bmatrix}
        1 \\ 0
      \end{bmatrix}, \qquad
      &&\frac{\partial \bW}{\partial p_6} &= \begin{bmatrix}
        0 \\ 1
      \end{bmatrix}
    \end{alignat*}
    Therefore, the Jacobian matrix $\bJ$ is:
    \[
      \blue{\bJ = \begin{bmatrix}
        x & 0 & y & 0 & 1 & 0 \\ 0 & x & 0 & y & 0 & 1
      \end{bmatrix}}
    \]
    
    \emph{
      \underline{Note}: Since the last row of $\bW$ is always $[0, 0, 1]$
      and therefore leaves the $x$-coordinate (in homogeneous coordinates)
      unchanged, we do not need to include the partial derivatives of the
      last row of $\bW$ with respect to the warp parameters $\bp$
      since the last row would be all zeros.
    }
  \end{answer}
\end{problem}

\newpage
\begin{problem}
  Find the computational complexity (Big O notation) for the initialization
  step (pre-computing $\bJ$ and $\bH^{-1}$) and for each runtime iteration
  (Equation 13) of the Matthews-Baker method:
  \[
    \Delta \bp^\ast = \bH^{-1} \bJ^\top \brackets{\bI(\bW(\bx; \bp)) - \bT}
  \]
  Express your answers in terms of $n$, $m$, and $p$, where
  \begin{enumroman}
    \item $n$ is the number of pixels in the template $\bT$,
    \item $m$ is the number of pixels in an input image $\bI$, and
    \item $p$ is the number of parameters used to describe the warp $\bW$.
  \end{enumroman}
  How does this compare to the runtime of the regular Lucas-Kanade method?
  \begin{answer}
    \begin{enumarabic}
      \item \textbf{Initialization Step}:
        \begin{enumalph}
          \item \textbf{Pre-computing $\bJ$}:
            The computational complexity of pre-computing $\bJ$ is
            $\blue{O(np)}$ since we need to compute the partial derivatives
            of the warp $\bW$ with respect to each of the warp parameters $\bp$,
            for each of the $n$ pixels in the image.
          \item \textbf{Pre-computing $\bH^{-1}$}:
            The computational complexity of pre-computing $\bH^{-1}$ is
            $\blue{O(p^3)}$ since we need to compute the inverse of the
            Hessian matrix $\bH$.
        \end{enumalph}
    \end{enumarabic}
  \end{answer}
\end{problem}
